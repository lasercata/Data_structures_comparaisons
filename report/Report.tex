\documentclass[a4paper, 12pt, twoside]{article}


%------------------------------------------------------------------------
%
% Author                :   Lasercata
% Last modification     :   2024.05.21
%
%------------------------------------------------------------------------

%---------Init {{{1
%------Lang
\usepackage[french]{babel}
%\usepackage[english]{babel}


%See https://github.com/lasercata/LaTeX_Templates for the file latex_base.sty
\input{latex_base.sty}


%------Circuitikz
%\usetikzlibrary{babel}             %Uncomment this to use circuitikz
%\usetikzlibrary{shapes.geometric}  % To draw triangles in trees
%\usepackage[european]{circuitikz}            %Electrical circuits drawing

%------Sections
%---To change section numbering :
% \renewcommand\thesection{\Roman{section}}
% \renewcommand\thesubsection{\arabic{subsection})}
% \renewcommand\thesubsubsection{\textit \alph{subsubsection})}

%---To start numbering sections from 0
% \setcounter{section}{-1}

%---To hide subsubsection from the table of contents (show with max depth of 2)
% \setcounter{tocdepth}{2}


%------File structure
\usepackage{forest} %For the file structure


%------Logo
% \setlogo %Comment to remove the logo
%}}}1

%------Title (with default LaTeX style)
%\title{}
%\author{}
%\date{\today}

%---------------------------------Begin Document
\begin{document}
    
    % Title {{{1
    \thetitle{Structures de données}{comparaisons de structures}
    %\maketitle
    
    \tableofcontents
    %\listoffigures
    %\listoftables
    %\listofalgorithms
    \newpage
    % }}}1
    
    \begin{indt}{\section{Méthode de développement}}% {{{1
        \begin{indt}{\subsection{Structure du projet}} %{{{2
            Voici comment sont organisés les fichiers du projet :
            \begin{figure}[H]% {{{4
                \centering
            
                \begin{tabular}{cc}
                    \begin{forest}% {{{5
                        for tree={
                            font=\sffamily,
                            text=black,
                            % text width=2cm,
                            % minimum height=0.75cm,
                            % if level=0
                            % {fill=ff4500}
                            % {draw=black},
                            rounded corners=4pt,
                            grow'=0,
                            child anchor=west,
                            parent anchor=south,
                            anchor=west,
                            calign=first,
                            edge={ff4500, rounded corners=1pt, line width=1pt},
                            edge path={
                                \noexpand\path [draw, \forestoption{edge}]
                                (!u.south west) +(0pt,0) |- (.child anchor)\forestoption{edge label};
                            },
                            before typesetting nodes={
                                if n=1
                                {insert before={[,phantom]}}
                                {}
                            },
                            fit=band,
                            s sep=1pt,
                            before computing xy={l=25pt},
                        }
                        [
                            [\textcolor{00f}{bin}]
                            [\textcolor{00f}{build}]
                            [\textcolor{00f}{data}]
                            [\textcolor{00f}{include}
                                [avl.h]
                                [bst.h]
                                [common.h]
                                [complexite.h]
                                [film.h]
                                [htbl.h]
                                [init\_random.h]
                                [liste.h]
                            ]
                            [\textcolor{00f}{logs}]
                            [\textcolor{00f}{plots}]
                            [\textcolor{00f}{report}]
                        ]
                    \end{forest}% }}}5
                    &
                    \begin{forest}% {{{5
                        for tree={
                            font=\sffamily,
                            text=black,
                            % text width=2cm,
                            % minimum height=0.75cm,
                            % if level=0
                            % {fill=ff4500}
                            % {draw=black},
                            rounded corners=4pt,
                            grow'=0,
                            child anchor=west,
                            parent anchor=south,
                            anchor=west,
                            calign=first,
                            edge={ff4500, rounded corners=1pt, line width=1pt},
                            edge path={
                                \noexpand\path [draw, \forestoption{edge}]
                                (!u.south west) +(0pt,0) |- (.child anchor)\forestoption{edge label};
                            },
                            before typesetting nodes={
                                if n=1
                                {insert before={[,phantom]}}
                                {}
                            },
                            fit=band,
                            s sep=1pt,
                            before computing xy={l=25pt},
                        }
                        [
                            [\textcolor{00f}{src}
                                [avl.c]
                                [bst.c]
                                [common.c]
                                [complexite.c]
                                [film.c]
                                [htbl.c]
                                [init\_random.c]
                                [liste.c]
                                [main.c]
                            ]
                            [make\_graphs.py]
                            [make\_logs.sh]
                            [Makefile]
                            [README.md]
                        ]
                    \end{forest}% }}}5
                \end{tabular}
            
                \caption{Arborescence du projet}
                \label{fig:forest}
            \end{figure}% }}}4

            Et voici à quoi correspondent les fichiers :
            
            \begin{table}[H]% {{{4
                \centering
            
                \begin{tabular}{|l|l|}
                    \hline
                    Description
                    & Fichiers
                    \\
                    \hline
                    \hline
                    Implémentation des \textbf{tables de hachage}
                    & \texttt{htbl.h}, \texttt{htbl.c}
                    \\
                    \hline
                    Implémentation des \textbf{listes} (pour \texttt{htbl})
                    & \texttt{liste.h}, \texttt{liste.c}
                    \\
                    \hline
                    Implémentation des \textbf{ABR}
                    & \texttt{bst.h}, \texttt{bst.c}
                    \\
                    \hline
                    Implémentation des \textbf{AVL}
                    & \texttt{avl.h}, \texttt{avl.c}
                    \\
                    \hline
                    \textbf{Test} d'une structure donnée pour un fichier test donné
                    & \texttt{main.c}
                    \\
                    \hline
                    Génération de \textbf{logs}
                    & \texttt{make\_logs.sh}
                    \\
                    \hline
                    Génération des \textbf{graphes}
                    & \texttt{make\_graphs.py}
                    \\
                    \hline
                \end{tabular}
            
                \caption{Description des fichiers}
                \label{tab:file_desc}
            \end{table}% }}}4
        \end{indt} %}}}2

        \begin{indt}{\subsection{Intégration}} %{{{3
            Chaque structure a été implémentée dans le TP correspondant, en utilisant le type \texttt{int}.
            Il a été nécessaire de faire quelques modifications afin de changer vers le type \texttt{t\_film} : changement dans la définition, et dans les fonctions de recherche (utilisation de \texttt{equals} pour le test d'égalité, de \texttt{le} pour $\le$).

            Pour créer les graphes, il suffit de faire \texttt{make graphs} (ou \texttt{make graphs SEARCH\_NB=[nombre de recherches à faire]}) : cela va compiler les sources en \texttt C ; lancer \texttt{make\_logs.sh} sur le résultat de cette compilation ; puis lancer \texttt{make\_graphs.py} sur les logs, ce qui va enregistrer les graphes dans \texttt{plots}.
        \end{indt} %}}}3
    \end{indt}% }}}1

    \vspace{12pt}
    
    \begin{indt}{\section{Résultats}} %{{{1
        .a
    \end{indt} %}}}1
    
\end{document}
%--------------------------------------------End

% vim:foldmethod=marker:foldlevel=0
